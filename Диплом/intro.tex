\section*{Введение}
\addcontentsline{toc}{section}{Введение}
Клеточные автоматы(КА) - вид дискретных математических моделей. Описывается любой КА указанием его элементов:
\begin{itemize}
\item Размерность и носитель решетки
\item Шаблон соседства
\item Множество состояний клеток
\item Функция перехода
\end{itemize}•
Модели на основе клеточных автоматов носят название Rule-based (или модели, основанные на правилах) в противоположность классическим моделям динамических систем, которые основаны на уравнениях (и называются, соответственно Equation-based)

\textbf{Цель} моей работы - показать возможности клеточно-автоматного подхода к решению различных классических задач математической физики. Выявить его преимущества и недостатки.

Считаю нужным здесь поместить общую классификацию клеточных автоматов, более формальные же математические выкладки я проведу в следующем разделе. Классический клеточный автомат имеет следующее словесное
 \begin{defn}
 КА – это регулярная структура двоичных конечных автоматов с одинаковыми правилами переходов, выраженных в виде булевых функций от состояний соседних автоматов 
 \end{defn}
 В дальнейшем клеточные автоматы стали рассматриваться в качестве более общих объектов, что породило их классификацию
 \begin{itemize}
\item По множеству состояний (алфавиту) клетки: булев, символьный, вещественный, целый и т.д.
\item По типу функции переходов - детерминированный/стохастический
\item По порождаемым им структурам (паттернам) - классификация Вольфрама
\item По режиму его работы - синхронный/асинхронный
\end{itemize}•

Для примера давайте рассмотрим простейший клеточный автомат называемый игрой Жизнь (Conway's Game of Life)
Пусть задана 2D-решетка (прямоугольная). Алфавит у этого автомата булев.
Окрестность каждой клетки задаем как окрестность Мура (считаем соседями клетки с общей стороной и углом)
Теперь определим функцию переходов следующим образом:
\begin{equation*}
c(t+1) = \begin{cases}
1, \text{если} \sum_k {c_k(t)} =3 \\
0, \text{иначе}
\end{cases}•
\end{equation*}•